\documentclass[spanish,a4paper,12pt]{article}
\usepackage{libertine}
\usepackage[T1]{fontenc}
\usepackage[utf8]{inputenc}
\usepackage{babel}
\usepackage{slantsc}
\usepackage{array}
\usepackage{amsmath}
\usepackage{amsfonts}
\usepackage{amssymb}
\usepackage{graphicx}
\usepackage{fancybox}
\usepackage{mathcomp}
\usepackage{array}
\usepackage{multicol}
%\usepackage{cite}
\usepackage{enumerate}
\usepackage{listings}
\usepackage{multicol}
\usepackage{biblatex} %Imports biblatex package
\addbibresource{Bibliografia.bib} %Import the bibliography file
\usepackage{amsmath, amssymb, amsthm}

%\setkomafont{subsection}{\usefont{T1}{fvm}{m}{n}}
%\setkomafont{section}{\usefont{T1}{fvs}{b}{n}\Large}
\setcounter{secnumdepth}{0}
\pagestyle{empty}

\usepackage{hyperref}
\usepackage{csquotes}

\newtheorem{defin}{Definición}
\newtheorem{exe}{Ejemplo}
\newtheorem{teo}{Teorema}

%   Definiciones que usa JLR en sus códigos TeX  %
\def\mcm{{\textsf{m.c.m.}}}
\def\Car{{\textsf{Car}}}
\def\Ring{{\mathcal R}}
\def\Z{{\mathbb Z}}
\def\Q{{\mathbb Q}}
\def\R{{\mathbb R}}
\def\C{{\mathbb C}}
\def\N{{\mathbb N}}
\def\F{{\mathbb F}}
\newcommand{\overbar}[1]{\mkern 1.5mu\overline{\mkern-1.5mu#1\mkern-1.5mu}\mkern 1.5mu}


\usepackage[left=3cm,right=3cm,top=2cm,bottom=3cm]{geometry}
%\setlength{\parskip}{\baselineskip} %Espacio entre párrafos
\setlength{\parindent}{12pt}

\title{Tarea 2}
\author{Victor Gandica, David Tellez}
\date{}

\begin{document}
\maketitle

\begin{enumerate}
  \item Sea $L$ un cuerpo. 
  \begin{enumerate}
    \item Sea $v$ una vaulación sobre $L$ y sea $O_v$ su anillo de valuación. Muestre que para todo $x \in L$ si $x \not\in O_v$ entonces $x^{-1} \in O_v$.
    \item Muestre que el converso del anterior se tiene. Es decir, muestre que si $O$ es un subanillo de $L$ tal que para todo $x \in L$, si $x \not \in O$ entonces $x^{-1} \in O$, entonces $O$ es el anillo de valuación de alguna valuación $v$ sobre $L$.
  \end{enumerate}

  \begin{center}
    \rule[2mm]{4.3cm}{0.5pt}
    \Ovalbox{\textbf{Solución}}
    \rule[2mm]{4.3cm}{0.5pt}
  \end{center}

  % \begin{enumerate}
  %   \item Sea $G$ un grupo abeliano totalmente ordenado y $v: L^{*} \rightarrow G$ una valuación discreta. Primero notamos que, como $v$ es un homomorfismo entre $L^*$ y $G$, $v(u^{-1}) = -v(u)$ para cualquier $u \in L*$.  
  % \end{enumerate}


  \item Sea $R$ un domino, $P \subseteq R$ un ideal primo y sea $L$ un cuerpo tal que $R \subseteq L$. El proposito de este problema es mostrar que existe una valuación $v$ sobre $L$ tal que si $O_v$ es el anillo de valuación y $M_v$ es su ideal maximal entonces $R \subseteq O_v$ y $R \cup M_v = P$
  
  \begin{enumerate}
    \item Muestre, que reemplazando $(R, P)$ por $(R_P , P_P)$ se puede asumir que $R$ es local con ideal maximal $P$.
    \item Considere la colección $\Sigma$ de parejas $(O, M)$ tales que $R \subseteq O \subseteq L$ es un anillo, $M \subseteq O$ un ideal propio y $P = R \cup M$. Muestre que $\Sigma$ puede ser dotado de un orden parcial, vía contenencia, y que $\Sigma$ tiene elementos maximales.
    \item Sea $(O, M)$ un elemento maximal de $\Sigma$. Muestre que $M$ es maximal. Más aún muestre que $O$ es local.
    \item Sea $(O, M)$ como en el punto anterior. Muestre que para todo $x \in L$ si $x \not \in O$, entonces $x^{-1} \in O$. Concluya del punto anterior el resultado.
  \end{enumerate}

  \begin{center}
    \rule[2mm]{4.3cm}{0.5pt}
    \Ovalbox{\textbf{Solución}}
    \rule[2mm]{4.3cm}{0.5pt}
  \end{center}
  
  \item Sea $K$ un cuerpo de números y $O_K$ su anillo de enteros.
  \begin{enumerate}
    \item Si $p \in Z$ un primo, muestre que existe un ideal maximal $M \subseteq O_K$ tal que $M \cup Z = pZ$.
    \item Suponga que la extensión $K/Q$ es de Galois con grupo de Galois $G$. Muestre que $O_K$ es invariante bajo $G$. Sea $p \in Z$ un primo y $\Sigma_p$ el conjunto de ideales maximales $M$ en $O_K$ tales que $p \in M$. Muestre que $G$ actua sobre el conjunto $\Sigma_p$.
    \item Sean $G, p$ y $\Sigma_p$ como en el inciso anterior y sea $M \in \Sigma_p$. Se definen el \textit{subgrupo de descomposición} de $M$ como
      \begin{center}
        $D_M := Stab_G (M)$
      \end{center}
    y el \textit{subgrupo de inercia} de M como
      \begin{center}
        $I_M := {\sigma \in G : \sigma(x) - x \in M, \forall x \in O_K}$
      \end{center}
    Muestre que $I_M \trianglelefteq D_M \leq G$ y que $D_M/I_M$ es un grupo cíclico.
    \item Suponga que $K = \Q(i)$. Para cada primo $p \in \Z$ y para cada $M \in \Sigma_p$ encuentre los grupos $I_M$ y $D_M$. ¿Para cuáles primos $p$ se tiene que $I_M$ es no trivial, y para cuáles $D_M$ es trivial?
  \end{enumerate}
\end{enumerate}


\end{document}